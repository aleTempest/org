% Created 2023-11-10 Fri 12:34
% Intended LaTeX compiler: pdflatex
\documentclass[11pt]{article}
\usepackage[utf8]{inputenc}
\usepackage[T1]{fontenc}
\usepackage{graphicx}
\usepackage{longtable}
\usepackage{wrapfig}
\usepackage{rotating}
\usepackage[normalem]{ulem}
\usepackage{amsmath}
\usepackage{amssymb}
\usepackage{capt-of}
\usepackage{hyperref}
\author{Agustín Alejandro Mota Hinojosa}
\date{\today}
\title{upv}
\hypersetup{
 pdfauthor={Agustín Alejandro Mota Hinojosa},
 pdftitle={upv},
 pdfkeywords={},
 pdfsubject={},
 pdfcreator={Emacs 29.1 (Org mode 9.7)}, 
 pdflang={English}}
\begin{document}

\maketitle
\tableofcontents

\section{Programación Orientada a Objetos}
\label{sec:orgba7a88a}

\subsection{Generar un projecto de javafx con maven}
\label{sec:orgf6a5f9d}
\href{https:openjfx.io/openjfx-docs/\#maven}{Getting Started with JavaFX: Run HelloWorld using Maven}

\begin{verbatim}
mvn archetype:generate \
        -DarchetypeGroupId=org.openjfx \
        -DarchetypeArtifactId=javafx-archetype-simple \
        -DarchetypeVersion=0.0.3 \
        -DgroupId=$group-id \
        -DartifactId=$name \
        -Dversion=1.0.0 \
        -Djavafx-version=21.0.1
\end{verbatim}
\subsection{{\bfseries\sffamily DONE} Colores}
\label{sec:org4eaf204}
Aplicación con botones que representen coloresy un objeto en medio. Al
presionar un botón el objeto se colorea del color respectivo.
\subsection{{\bfseries\sffamily DONE} Eventos de interaz}
\label{sec:orgaffaee5}
Aplicación que implemente eventos y sus respectivos métodos.
\subsection{{\bfseries\sffamily DONE} Generador de contraseñas}
\label{sec:org3a3dd28}
Fuentes:
\begin{itemize}
\item \href{https:stackoverflow.com/questions/31260512/generate-a-secure-random-password-in-java-with-minimum-special-character-require}{Generate a Secure Random Password in Java with Minimum Special Character
Requirements}
\end{itemize}
\begin{quote}
El boton generara una contraseña aleatoria
nosotros le daremos click en generar contraseña y se generara aleatoria y
la contraseña se mostrara en el textil

Definir un ancho o tamaño total de contraseña

Cuantos bytes: max 10 bytes

La contraseña uncluye todas las letras del abecedario, simbolos y signos,
caracteres
\end{quote}
\subsection{{\bfseries\sffamily DONE} Entregar actividad de pase de lista}
\label{sec:org9dbd0a6}
\subsection{{\bfseries\sffamily DONE} Números perfectos}
\label{sec:orge2187d3}
Aplicación para generar números perfectos de 4-5.
\subsection{{\bfseries\sffamily DONE} Snake}
\label{sec:org907cddc}
Un juego de la serpiente dentro de una matriz de 50x50.
\subsection{{\bfseries\sffamily DONE} Notas}
\label{sec:org8f509b0}
Aplicación para guardar notas
SCHEDULED: \textit{<2023-10-13 Fri>}
\subsection{{\bfseries\sffamily DONE} Galería de imagenes}
\label{sec:orgd178174}
Mostrar un carrusel de imagenes, la central va a ser la más grande y las
demás (al lado) serán más chicas. Con un rollbar, seleccionar el tiempo
con el que se va a cambiar la imagen. 10 imagenes
\subsection{{\bfseries\sffamily NO} Aplicación Feria}
\label{sec:orgc5f94b5}
\subsubsection{Notas}
\label{sec:orgd04101d}
\begin{itemize}
\item[{$\square$}] Excepciones para la validación de los campos
\begin{itemize}
\item[{$\square$}] Validar Email
\item[{$\boxtimes$}] Validar usuario o email duplicados
\item[{$\square$}] Validar contraseña
\end{itemize}
\item[{$\square$}] Crear nueva branch y eliminar cambios de la demo
\end{itemize}
\subsubsection{{\bfseries\sffamily DONE} DEMO Login funcional y Pantallas [6/6]}
\label{sec:orga113f91}
\begin{itemize}
\item[{$\boxtimes$}] Login
\item[{$\boxtimes$}] Registro
\item[{$\boxtimes$}] Pantalla principal
\item[{$\boxtimes$}] Eventos
\item[{$\boxtimes$}] Entradas
\item[{$\boxtimes$}] Comida
\end{itemize}
SCHEDULED: \textit{<2023-10-20 Fri>}
\subsection{{\bfseries\sffamily DONE} Login con BD}
\label{sec:org8abadde}
Login básico con usuario y contraseña
\subsubsection{Base de datos del login}
\label{sec:org8e02f81}
\begin{enumerate}
\item Script
\label{sec:orgf691ca2}
\begin{verbatim}
CREATE TABLE users (
       id           number primary key,
       first_name     varchar(30),
       last_name     varchar(40),
       email     varchar(40),
       phone_number     varchar(30)
);

INSERT INTO users VALUES (1,'user1','123');
INSERT INTO users VALUES (2,'user2','123');
INSERT INTO users VALUES (3,'user3','123');
INSERT INTO users VALUES (4,'user4','123');
\end{verbatim}
\end{enumerate}
\subsection{{\bfseries\sffamily NO} Evaluación Segunda Unidad}
\label{sec:org65fe474}
\subsection{Unida 3 Trabajo 1}
\label{sec:orgc312c61}
\section{Base de datos}
\label{sec:orgdf95749}
\url{https://oracleacademysolved.blogspot.com/}
\subsection{{\bfseries\sffamily DONE} Bloque 1 [3/3]}
\label{sec:orgea7fa97}
SCHEDULED: \textit{<2023-10-20 Fri>}
\begin{itemize}
\item[{$\boxtimes$}] Práctica 1-1
\item[{$\boxtimes$}] Práctica 1-2
\item[{$\boxtimes$}] Práctica 1-3
\end{itemize}
\subsection{{\bfseries\sffamily DONE} Bloque 2 [3/3]}
\label{sec:org79a87db}
\begin{itemize}
\item[{$\boxtimes$}] Práctica 2-1
\item[{$\boxtimes$}] Práctica 2-2
\item[{$\boxtimes$}] Práctica 2-3
\item[{$\boxtimes$}] Práctica 2-4
\end{itemize}
\subsubsection{{\bfseries\sffamily DONE} Práctica 2-6}
\label{sec:org8a94d33}
\subsubsection{{\bfseries\sffamily DONE} Práctica 2-7}
\label{sec:org700ce95}
\subsubsection{{\bfseries\sffamily DONE} QUIZ}
\label{sec:org258b02a}
\subsection{{\bfseries\sffamily DONE} Bloque 3}
\label{sec:org2ddc4ff}
\subsubsection{{\bfseries\sffamily DONE} Práctica 3-1}
\label{sec:orgfadcf0d}
\subsubsection{{\bfseries\sffamily DONE} Práctica 3-2}
\label{sec:orgb2cfad2}
\subsubsection{{\bfseries\sffamily DONE} Práctica 3-3}
\label{sec:org94db5e3}
\subsubsection{{\bfseries\sffamily DONE} Práctica 3-4}
\label{sec:org31a03c0}
\subsubsection{{\bfseries\sffamily DONE} Quizz Bloque 3}
\label{sec:orgda94008}
\subsection{{\bfseries\sffamily TODO} Bloque 4}
\label{sec:org9780baa}
\begin{itemize}
\item[{$\square$}] Practice PL/SQL 4-1
\item[{$\square$}] Practice PL/SQL 4-2
\item[{$\square$}] Practice PL/SQL 4-3
\item[{$\square$}] Practice PL/SQL 4-4
\item[{$\square$}] Practice PL/SQL 4-5
\end{itemize}
\section{Ecuaciones Diferenciales}
\label{sec:org7ceffb5}
\subsection{{\bfseries\sffamily DONE} Formulario de ED}
\label{sec:orgbec7b1c}
\subsection{{\bfseries\sffamily NO} Actividad 1: Método de reducción}
\label{sec:org0cd3c36}
:SCORE: 0
\subsection{{\bfseries\sffamily DONE} Actividad 2}
\label{sec:orgb660a9d}
CLOSED: \textit{<2023-10-20 Fri>}
:SCORE: 1
\subsection{{\bfseries\sffamily DONE} Actividad 3}
\label{sec:org851411b}
:SCORE: 1
SCHEDULED: \textit{<2023-10-16 Mon>}
CLOSED: \textit{<2023-10-16 Mon>}
\subsection{{\bfseries\sffamily DONE} Actividad 4}
\label{sec:org5f65adf}
:SCORE: 1
SCHEDULED: \textit{<2023-10-18 Wed>}
\subsection{{\bfseries\sffamily NO} Actividad 5}
\label{sec:orgc0d29fc}
\subsection{{\bfseries\sffamily DONE} Reporte}
\label{sec:orgf570d63}
\subsection{{\bfseries\sffamily DONE} Actividad en equipo}
\label{sec:org856f051}
\subsection{{\bfseries\sffamily DONE} Examen Mate Unidad 2}
\label{sec:org473fe0b}
\subsection{{\bfseries\sffamily NO} Actividad 1 Unidad 3}
\label{sec:orgd0d81c2}
:SCORE: 0
\subsection{{\bfseries\sffamily TODO} Actividad 2 Unidad 3}
\label{sec:orgc9b8cae}
:SCORE:
\section{Inglés}
\label{sec:org897bba9}
\subsection{{\bfseries\sffamily DONE} Presentación Unidad 2}
\label{sec:orga8e90bd}
CLOSED: \textit{<2023-10-17 Tue>}
\subsection{{\bfseries\sffamily DONE} Examen Unidad 2}
\label{sec:org6df2052}
\end{document}