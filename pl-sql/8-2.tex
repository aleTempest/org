% Created 2023-12-03 dom 20:06
% Intended LaTeX compiler: pdflatex
\documentclass[11pt]{article}
\usepackage[utf8]{inputenc}
\usepackage[T1]{fontenc}
\usepackage{graphicx}
\usepackage{longtable}
\usepackage{wrapfig}
\usepackage{rotating}
\usepackage[normalem]{ulem}
\usepackage{amsmath}
\usepackage{amssymb}
\usepackage{capt-of}
\usepackage{hyperref}
\usepackage[margin=0.5in]{geometry}
\author{Agustín Alejandro Mota Hinojosa}
\date{\today}
\title{8-2}
\hypersetup{
 pdfauthor={Agustín Alejandro Mota Hinojosa},
 pdftitle={8-2},
 pdfkeywords={},
 pdfsubject={},
 pdfcreator={Emacs 29.1 (Org mode 9.7)}, 
 pdflang={English}}
\begin{document}

\maketitle
\tableofcontents

\section{Vocabulary}
\label{sec:orgad1d907}
\begin{enumerate}
\item Parameter Passing: Pass or communicate data between the caller and subprogram.
\item Actual Parameter: The actual value assigned to a parameter.
\item Actual Argument: Can be literal values, variables, or expressions that are provided in the parameter list of a called subprogram.
\item Formal Parameter: A parameter name declared in the procedure heading.
\end{enumerate}
\section{Try}
\label{sec:orgb3ce8b8}
\begin{enumerate}
\item Parameters in PL/SQL subprograms are variables used to pass values into or out of subprograms such as procedures and functions.
\begin{itemize}
\item There are two types of parameters:
\begin{itemize}
\item IN Parameters: Used for passing values into the subprogram. The values are read-only within the subprogram.
\item OUT Parameters: Used for passing values out of the subprogram. The subprogram can modify the values of OUT parameters.
\item IN OUT Parameters: Used for both passing values into and out of the subprogram.
\end{itemize}
\end{itemize}
\item COUNTRIES table:
\begin{enumerate}
\item code:
\begin{verbatim}
CREATE OR REPLACE PROCEDURE get_country_info(
   p_country_id IN COUNTRIES.COUNTRY_ID%TYPE
) AS
   v_country_name COUNTRIES.COUNTRY_NAME%TYPE;
   v_capital COUNTRIES.CAPITAL%TYPE;
BEGIN
   SELECT COUNTRY_NAME, CAPITAL INTO v_country_name, v_capital
   FROM COUNTRIES
   WHERE COUNTRY_ID = p_country_id;

   DBMS_OUTPUT.PUT_LINE('Country: ' || v_country_name || ', Capital: ' || v_capital);
EXCEPTION
   WHEN NO_DATA_FOUND THEN
      DBMS_OUTPUT.PUT_LINE('No information found for Country ID ' || p_country_id);
END get_country_info;
/
\end{verbatim}

\item execute:
\begin{verbatim}
BEGIN
   get_country_info(90);
END;
\end{verbatim}

\item re-execute: This would display the information for the country with ID 95 if it exists; otherwise, it will handle the NO\textsubscript{DATA}\textsubscript{FOUND} exception.

\item Modify
\begin{verbatim}
CREATE OR REPLACE PROCEDURE get_country_info(
   p_country_id IN COUNTRIES.COUNTRY_ID%TYPE
) AS
   v_country_name COUNTRIES.COUNTRY_NAME%TYPE;
   v_capital COUNTRIES.CAPITAL%TYPE;
BEGIN
   BEGIN
      SELECT COUNTRY_NAME, CAPITAL INTO v_country_name, v_capital
      FROM COUNTRIES
      WHERE COUNTRY_ID = p_country_id;

      DBMS_OUTPUT.PUT_LINE('Country: ' || v_country_name || ', Capital: ' || v_capital);
   EXCEPTION
      WHEN NO_DATA_FOUND THEN
         DBMS_OUTPUT.PUT_LINE('No information found for Country ID ' || p_country_id);
   END;
END get_country_info;
/
\end{verbatim}
\end{enumerate}

\item Formal Parameter: Parameters declared in the subprogram specification or body. They define the interface of the subprogram.

\item Procedure for Region's Highest Elevation:

\begin{enumerate}
\item code:
\begin{verbatim}
CREATE OR REPLACE PROCEDURE country_count_by_elevation(
   p_region_id IN REGIONS.REGION_ID%TYPE,
   p_min_elevation IN NUMBER,
   p_first_char IN CHAR DEFAULT NULL
) AS
   v_count NUMBER;
BEGIN
   SELECT COUNT(*)
   INTO v_count
   FROM COUNTRIES c
   JOIN LOCATIONS l ON c.LOCATION_ID = l.LOCATION_ID
   WHERE c.REGION_ID = p_region_id
      AND l.HIGHEST_ELEVATION > p_min_elevation
      AND (p_first_char IS NULL OR SUBSTR(c.COUNTRY_NAME, 1, 1) = p_first_char);

   DBMS_OUTPUT.PUT_LINE('Number of countries: ' || v_count);
END country_count_by_elevation;
/
\end{verbatim}

\item Execute:
\begin{verbatim}
BEGIN
   country_count_by_elevation(5, 2000);
END;
\end{verbatim}

\item DESC table: Can't because I'm not in apex

\item modify code:
\begin{verbatim}
CREATE OR REPLACE PROCEDURE country_count_by_elevation(
   p_region_id IN REGIONS.REGION_ID%TYPE,
   p_min_elevation IN NUMBER,
   p_first_char IN CHAR DEFAULT NULL
) AS
   v_count NUMBER;
BEGIN
   SELECT COUNT(*)
   INTO v_count
   FROM COUNTRIES c
   JOIN LOCATIONS l ON c.LOCATION_ID = l.LOCATION_ID
   WHERE c.REGION_ID = p_region_id
      AND l.HIGHEST_ELEVATION > p_min_elevation
      AND (p_first_char IS NULL OR SUBSTR(c.COUNTRY_NAME, 1, 1) = p_first_char);

   DBMS_OUTPUT.PUT_LINE('Number of countries: ' || v_count);
EXCEPTION
   WHEN NO_DATA_FOUND THEN
      DBMS_OUTPUT.PUT_LINE('No countries found for the specified criteria.');
END country_count_by_elevation;
/
\end{verbatim}

\item Anonymous Block

\begin{verbatim}
   DECLARE
      v_region_id REGIONS.REGION_ID%TYPE := 5;
      v_elevation NUMBER := 2000;
      v_first_char CHAR := 'B';
   BEGIN
      country_count_by_elevation(v_region_id, v_elevation, v_first_char);
   END;
\end{verbatim}

\item code
\begin{verbatim}
DECLARE
   v_region_id REGIONS.REGION_ID%TYPE := 5;
   v_elevation NUMBER := 2000;
   v_first_char CHAR := 'B';
BEGIN
   country_count_by_elevation(v_first_char, v_region_id, v_elevation);
END;
\end{verbatim}
\end{enumerate}
\end{enumerate}
\end{document}