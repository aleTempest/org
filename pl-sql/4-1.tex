% Created 2023-11-10 Fri 22:35
% Intended LaTeX compiler: pdflatex
\documentclass[11pt]{article}
\usepackage[utf8]{inputenc}
\usepackage[T1]{fontenc}
\usepackage{graphicx}
\usepackage{longtable}
\usepackage{wrapfig}
\usepackage{rotating}
\usepackage[normalem]{ulem}
\usepackage{amsmath}
\usepackage{amssymb}
\usepackage{capt-of}
\usepackage{hyperref}
\author{Agustín Alejandro Mota Hinojosa}
\date{\today}
\title{Database Programming with PL/SQL}
\hypersetup{
 pdfauthor={Agustín Alejandro Mota Hinojosa},
 pdftitle={Database Programming with PL/SQL},
 pdfkeywords={},
 pdfsubject={},
 pdfcreator={Emacs 29.1 (Org mode 9.7)}, 
 pdflang={English}}
\begin{document}

\maketitle
\tableofcontents

\section{Vocabulary}
\label{sec:org2a690f4}

Statement that enables PL/SQL to perform actions selectively based on conditions.

\textbf{IF}

Control structures – Repetition statements that enable you to execute statements in a PL/SQL block repeatedly.

\textbf{LOOP}

An expression with a TRUE or FALSE value that is used to make a decision.

\textbf{Condition}

An expression that determines a course of action based on conditions and can be used outside a PL/SQL block in a SQL statement.

\textbf{CASE}
\section{Try it / solve it}
\label{sec:org0c9e4e8}

\begin{enumerate}
\item What is the purpose of a conditional control structure in PL/SQL?
\end{enumerate}


The purpose of a conditional control structure is to analyse variables and choose a direction in which to go based on given parameters.

\begin{enumerate}
\item List the three categories of control structures in PL/SQL.

\begin{itemize}
\item \texttt{IF} conditional constructors
\item \texttt{CASE} expressions
\item \texttt{LOOP} control structures
\end{itemize}

\item List the keywords that can be part of an IF statement.

\item \texttt{IF}
\item \texttt{END IF}
\item \texttt{ELSIF}
\item \texttt{ELSE}
\item \texttt{THEN}

\item List the keywords that are a required part of an IF statement.

\item \texttt{IF}
\item \texttt{THEN}
\item \texttt{END IF}

\item Write a PL/SQL block to find the population of a given country in the countries table. Display a

\item message indicating whether the population is greater than or less than 1 billion (1,000,000,000).
\item Test your block twice using India (country\textsubscript{id} = 91) and United Kingdom (country\textsubscript{id} = 44). India’s
\item population should be greater than 1 billion, while United Kingdom’s should be less than 1 billion.
\end{enumerate}


\begin{verbatim}
declare
  v_populatie wf_countries.population%TYPE;
begin
  select population into v_populatie
  from wf_countries
  where country_id=91;
  
  dbms_output.put_line(v_populatie);
  
  if v_populatie > 1000000000 then
    dbms_output.put_line('Mai mare de 1 miliard');
  else
    dbms_output.put_line('Mai mica de 1 miliard');
  end if;
  
  select population into v_populatie
  from wf_countries
  where country_id=44;
  
  dbms_output.put_line(v_populatie);
  
  if v_populatie > 1000000000 then
    dbms_output.put_line('Mai mare de 1 miliard');
  else
    dbms_output.put_line('Mai mica de 1 miliard');
  end if;
end;

\end{verbatim}

\begin{enumerate}
\item Modify the code from the previous exercise so that it handles all the following cases:

\begin{enumerate}
\item Population is greater than 1 billion.

\item Population is greater than 0.

\item Population is 0.
\end{enumerate}

\item Population is null. (Display: No data for this country.)
\end{enumerate}

Run your code using the following country ids. Confirm the indicated results.

\begin{itemize}
\item China (country\textsubscript{id} = 86): Population is greater than 1 billion.

\item United Kingdom (country\textsubscript{id} = 44): Population is greater than 0.

\item Antarctica (country\textsubscript{id} = 672): Population is 0.

\item Europa Island (country\textsubscript{id} = 15): No data for this country.
\end{itemize}


\begin{verbatim}
declare
  v_populatie wf_countries.population%TYPE;

begin
  select population into v_populatie
  from wf_countries
  where country_id=86;
  
  dbms_output.put_line(v_populatie);

  if v_populatie > 1000000000 then
    dbms_output.put_line('Population is greater than 1 billion');
  elsif v_populatie = 0 then
    dbms_output.put_line('Population is 0');
  elsif v_populatie <= 1000000000 then
    dbms_output.put_line('Population is greater than 0');
  else
    dbms_output.put_line('Population is null');
  end if;

  select population into v_populatie
  from wf_countries
  where country_id=44;

  dbms_output.put_line(v_populatie);

  if v_populatie > 1000000000 then
    dbms_output.put_line('Population is greater than 1 billion');
  elsif v_populatie = 0 then
    dbms_output.put_line('Population is 0');
  elsif v_populatie <= 1000000000 then
    dbms_output.put_line('Population is greater than 0');
  else
    dbms_output.put_line('Population is null');
  end if;

  select population into v_populatie
  from wf_countries
  where country_id=672;

  dbms_output.put_line(v_populatie);

  if v_populatie > 1000000000 then
    dbms_output.put_line('Population is greater than 1 billion');
  elsif v_populatie = 0 then
    dbms_output.put_line('Population is 0');
  elsif v_populatie <= 1000000000 then
    dbms_output.put_line('Population is greater than 0');
  else
    dbms_output.put_line('Population is null');
  end if;

  select population into v_populatie
  from wf_countries
  where country_id=15;

  dbms_output.put_line(v_populatie);

  if v_populatie > 1000000000 then
    dbms_output.put_line('Population is greater than 1 billion');
  elsif v_populatie = 0 then
    dbms_output.put_line('Population is 0');
  elsif v_populatie <= 1000000000 then
    dbms_output.put_line('Population is greater than 0');
  else
    dbms_output.put_line('Population is null');
  end if;
end;
\end{verbatim}


\begin{enumerate}
\item Examine the following code:
\end{enumerate}

\begin{verbatim}
DECLARE
  v_country_id countries.country_name%TYPE := 'ABC';
  v_ind_date countries.date_of_independence%TYPE;
  v_natl_holiday countries.national_holiday_date%TYPE;
BEGIN
  SELECT date_of_independence, national_holiday_date
  INTO v_ind_date, v_natl_holiday
  FROM countries
  WHERE country_id = v_country_id;

  IF v_ind_date IS NOT NULL THEN
    DBMS_OUTPUT.PUT_LINE('A');
  ELSIF v_natl_holiday IS NOT NULL THEN
    DBMS_OUTPUT.PUT_LINE('B');
  ELSIF v_natl_holiday IS NULL AND v_ind_date IS NULL THEN
    DBMS_OUTPUT.PUT_LINE('C');
  END IF;
END;
\end{verbatim}

\begin{enumerate}
\item What would print if the country has an independence date equaling NULL and a national

\textbf{holiday date equaling NULL?}

\item What would print if the country has an independence date equaling NULL and a national

\textbf{holiday date containing a value?}

\item What would print if the country has an independence date equaling a value and a national

\textbf{holiday date equaling NULL?}

\item Run a SELECT statement against the COUNTRIES table to determine whether the following

\item countries have independence dates or national holiday dates, or both. Predict the output of
\item running the anonymous block found at the beginning of this question.
\item Country - Country\textsubscript{ID} - IndependenceDate - National Holiday Date - Output should be
\item Antarctica - 672 - NO - NO - C
\item Iraq - 964 - Yes - No - A
\item Spain - 34 - NO - Yes - B
\item United States - 1 - Yes - No - A

\item Finally, run the anonymous block found at the beginning of this question using each of the
\end{enumerate}
above country ids as input. Check whether your output answers are correct.

\begin{verbatim}
DECLARE
  v_ind_date wf_countries.date_of_independence%TYPE;
  v_natl_holiday wf_countries.national_holiday_date%TYPE;
BEGIN
  SELECT date_of_independence, national_holiday_date
  INTO v_ind_date, v_natl_holiday
  FROM wf_countries
  WHERE country_id = 672;

  IF v_ind_date IS NOT NULL THEN
    DBMS_OUTPUT.PUT_LINE('Have independence Date');
    DBMS_OUTPUT.PUT_LINE('No national holiday date');
    DBMS_OUTPUT.PUT_LINE('A');
  ELSIF v_natl_holiday IS NOT NULL THEN
    DBMS_OUTPUT.PUT_LINE('No independence Date');
    DBMS_OUTPUT.PUT_LINE('Have national holiday date');
    DBMS_OUTPUT.PUT_LINE('B');
  ELSIF v_natl_holiday IS NULL AND v_ind_date IS NULL THEN
    DBMS_OUTPUT.PUT_LINE('No independence Date');
    DBMS_OUTPUT.PUT_LINE('No national holiday date');
    DBMS_OUTPUT.PUT_LINE('C');
  END IF;

  SELECT date_of_independence, national_holiday_date
  INTO v_ind_date, v_natl_holiday
  FROM wf_countries
  WHERE country_id = 964;

  IF v_ind_date IS NOT NULL THEN
    DBMS_OUTPUT.PUT_LINE('Have independence Date');
    DBMS_OUTPUT.PUT_LINE('No national holiday date');
    DBMS_OUTPUT.PUT_LINE('A');
  ELSIF v_natl_holiday IS NOT NULL THEN
    DBMS_OUTPUT.PUT_LINE('No independence Date');
    DBMS_OUTPUT.PUT_LINE('Have national holiday date');
    DBMS_OUTPUT.PUT_LINE('B');
  ELSIF v_natl_holiday IS NULL AND v_ind_date IS NULL THEN
    DBMS_OUTPUT.PUT_LINE('No independence Date');
    DBMS_OUTPUT.PUT_LINE('No national holiday date');
    DBMS_OUTPUT.PUT_LINE('C');
  END IF;

  SELECT date_of_independence, national_holiday_date
  INTO v_ind_date, v_natl_holiday
  FROM wf_countries
  WHERE country_id = 34;

  IF v_ind_date IS NOT NULL THEN
    DBMS_OUTPUT.PUT_LINE('Have independence Date');
    DBMS_OUTPUT.PUT_LINE('No national holiday date');
    DBMS_OUTPUT.PUT_LINE('A');
  ELSIF v_natl_holiday IS NOT NULL THEN
    DBMS_OUTPUT.PUT_LINE('No independence Date');
    DBMS_OUTPUT.PUT_LINE('Have national holiday date');
    DBMS_OUTPUT.PUT_LINE('B');
  ELSIF v_natl_holiday IS NULL AND v_ind_date IS NULL THEN
    DBMS_OUTPUT.PUT_LINE('No independence Date');
    DBMS_OUTPUT.PUT_LINE('No national holiday date');
    DBMS_OUTPUT.PUT_LINE('C');
  END IF;

  SELECT date_of_independence, national_holiday_date
  INTO v_ind_date, v_natl_holiday
  FROM wf_countries
  WHERE country_id = 1;

  IF v_ind_date IS NOT NULL THEN
    DBMS_OUTPUT.PUT_LINE('Have independence Date');
    DBMS_OUTPUT.PUT_LINE('No national holiday date');
    DBMS_OUTPUT.PUT_LINE('A');
  ELSIF v_natl_holiday IS NOT NULL THEN
    DBMS_OUTPUT.PUT_LINE('No independence Date');
    DBMS_OUTPUT.PUT_LINE('Have national holiday date');
    DBMS_OUTPUT.PUT_LINE('B');
  ELSIF v_natl_holiday IS NULL AND v_ind_date IS NULL THEN
    DBMS_OUTPUT.PUT_LINE('No independence Date');
    DBMS_OUTPUT.PUT_LINE('No national holiday date');
    DBMS_OUTPUT.PUT_LINE('C');
  END IF;
END;
\end{verbatim}

\begin{enumerate}
\item Examine the following code. What output do you think it will produce?

\begin{verbatim}
DECLARE
  v_num1 NUMBER(3) := 123;
  v_num2 NUMBER;
BEGIN
  IF v_num1 <> v_num2 THEN
    DBMS_OUTPUT.PUT_LINE('The two numbers are not equal');
  ELSE
    DBMS_OUTPUT.PUT_LINE('The two numbers are equal');
  END IF;
END;

\end{verbatim}

\item Write a PL/SQL block to accept a year and check whether it is a leap year. For example, if the
\end{enumerate}
year entered is 1990, the output should be “1990 is not a leap year.”
Hint: A leap year should be exactly divisible by 4, but not exactly divisible by 100. However, any
year exactly divisible by 400 is a leap year.
Test your solution with the following years:

Year Result Should Be

\begin{itemize}
\item 1990 Not a leap year
\item 2000 Leap year
\item 1996 Leap year
\item 1900 Not a leap year
\item 2016 Leap year
\item 1884 Leap year

\begin{verbatim}
DECLARE
  v_year NUMBER(20) := 1990;
BEGIN
  IF MOD(v_year, TO_NUMBER(400)) = 0 THEN
    DBMS_OUTPUT.PUT_LINE('Leap year');
  ELSIF MOD(v_year, TO_NUMBER(100)) = 0 THEN
    DBMS_OUTPUT.PUT_LINE('Not a leap year');
  ELSIF MOD(v_year, TO_NUMBER(4)) = 0 THEN
    DBMS_OUTPUT.PUT_LINE('Leap year');
  ELSE
    DBMS_OUTPUT.PUT_LINE('Not a leap year');
  END IF;
END;
\end{verbatim}
\end{itemize}
\end{document}