% Created 2023-10-20 vie 22:52
% Intended LaTeX compiler: pdflatex
\documentclass[11pt]{article}
\usepackage[utf8]{inputenc}
\usepackage[T1]{fontenc}
\usepackage{graphicx}
\usepackage{longtable}
\usepackage{wrapfig}
\usepackage{rotating}
\usepackage[normalem]{ulem}
\usepackage{amsmath}
\usepackage{amssymb}
\usepackage{capt-of}
\usepackage{hyperref}
\usepackage{minted}
\usepackage[margin=0.5in]{geometry}
\usepackage[spanish, english]{babel}
\author{Agustín Alejandro Mota Hinojosa}
\date{\today}
\title{2-3 Recognizing Data Types}
\hypersetup{
 pdfauthor={Agustín Alejandro Mota Hinojosa},
 pdftitle={2-3 Recognizing Data Types},
 pdfkeywords={},
 pdfsubject={},
 pdfcreator={Emacs 29.1 (Org mode 9.7)}, 
 pdflang={English}}
\begin{document}

\maketitle
\tableofcontents

\section{Vocabulary}
\label{sec:orgec26e06}
\begin{enumerate}
\item Store large blocks of single-byte or fixed-width multi-byte NCHAR data in the database.

\textbf{NCLOB}

\item Hold values, called locators, that specify the location of large objects (such as graphic images) that are stored out of line.

\textbf{LOB}

\item Hold a single value with no internal components.

\textbf{Scalar}

\item Store large unstructured or structured binary objects.

\textbf{BLOB}

\item Contain internal elements that are either scalar (record) or composite (record and table).

\textbf{Composite}

\item Store large binary files outside of the database.

\textbf{BFILE}

\item Hold values, called pointers, that point to a storage location.

\textbf{Reference}

\item A schema object with a name, attributes, and methods.

\textbf{Object}

\item Store large blocks of character data in the database.

\textbf{CLOB}
\end{enumerate}
\section{Try It / Solve It}
\label{sec:org4d6bf98}
\begin{enumerate}
\item In your own words, describe what a data type is and explain why it is important.

A data type is a classification or category that specifies which type of value a variable or column can hold. It defines the format and the operations that can be performed on the data.

\item Identify the three data type categories covered in this course.

\begin{enumerate}
\item LOB
\item Scalar
\item Composite
\end{enumerate}

\item Identify three data types covered in the Database Programming with SQL course.
\begin{enumerate}
\item Number
\item Date
\item Varchar2
\end{enumerate}

\item What data type can be used in PL/SQL, but can’t be used to define a table column?

\textbf{Boolean}

\item Which data type indicates a large data object that is stored outside of the database?

\textbf{BFILE}

\item Identify the data type category (LOB, Scalar, or Composite) for each data type. Each category may be used more than once.
\begin{center}
\begin{tabular}{ll}
Data Type & Data Type Category\\[0pt]
\hline
CLOB & LOB\\[0pt]
VARCHAR2 & Scalar\\[0pt]
BLOB & LOB\\[0pt]
NUMBER & Scalar\\[0pt]
BFILE & LOB\\[0pt]
TIMESTAMP & Scalar\\[0pt]
NCLOB & LOB\\[0pt]
RECORD & Composite\\[0pt]
PLS\textsubscript{INTEGER} & Scalar\\[0pt]
LONG & Scalar\\[0pt]
TABLE & Composite\\[0pt]
BOOLEAN & Scalar\\[0pt]
\end{tabular}
\end{center}

\item Enter the data type category and the data type for each value.
\begin{enumerate}
\item ‘Switzerland’
Scalar \texttt{VARCHAR2}
\item Text of a resume
Scalar \texttt{VARCHAR2}
\item 100.20
Scalar \texttt{NUMBER}
\item A picture
\texttt{LOB BLOB}
\item 1053
Scalar \texttt{NUMBER}
\item 11-Jun-2016
Scalar \texttt{DATE}
\item ‘Computer science is the science of the 21 st century.’
Scalar \texttt{VARCHAR2}
\item Following table:
\begin{center}
\begin{tabular}{rl}
Index & Last\textsubscript{name}\\[0pt]
\hline
1 & 'Newman'\\[0pt]
2 & 'Raman'\\[0pt]
3 & 'Han'\\[0pt]
\end{tabular}
\end{center}

Composite Table

\item A movie
\texttt{LOB BFILE}
\item A sound byte
\texttt{LOB BFILE}
\item FALSE
Scalar \texttt{BLOB}
\end{enumerate}
\end{enumerate}
\end{document}