% Created 2023-11-26 Sun 13:31
% Intended LaTeX compiler: pdflatex
\documentclass[11pt]{article}
\usepackage[utf8]{inputenc}
\usepackage[T1]{fontenc}
\usepackage{graphicx}
\usepackage{longtable}
\usepackage{wrapfig}
\usepackage{rotating}
\usepackage[normalem]{ulem}
\usepackage{amsmath}
\usepackage{amssymb}
\usepackage{capt-of}
\usepackage{hyperref}
\usepackage[margin=0.5in]{geometry}
\author{Agustín Alejandro Mota Hinojosa}
\date{\today}
\title{PL/SQL 6-1}
\hypersetup{
 pdfauthor={Agustín Alejandro Mota Hinojosa},
 pdftitle={PL/SQL 6-1},
 pdfkeywords={},
 pdfsubject={},
 pdfcreator={Emacs 29.1 (Org mode 9.7)}, 
 pdflang={English}}
\begin{document}

\maketitle
\tableofcontents

\section{Vocabulary}
\label{sec:orgef2286f}
\begin{enumerate}
\item Explicit Cursor
\item Cursor
\item CLOSE
\item Oracle Server
\item Implicit cursor
\item OPEN
\item FETCH
\item Active set
\end{enumerate}
\section{Try/Solve it}
\label{sec:org7f0a3cd}
\begin{enumerate}
\item An implicit cursor is defined by Oracle Server when you do DML statements and
a explicit cursor is a user declared for queries that return one row.
\item The SELECT statement
\item Two circumstances when you need an explicit cursor:
\begin{enumerate}
\item When you have a statement that returns more than one row.
\item To produce a report of all employees
\end{enumerate}
\item CURRENCIES tables:
\begin{enumerate}
\item Display all the rows in CURRENCIES:
\begin{verbatim}
DECLARE
    CURSOR currencies_cur IS SELECT currency_code, currency_name FROM WF_CURRENCIES ORDER BY currency_name;
    v_curr_name WF_CURRENCIES.currency_name%TYPE;
    v_curr_code WF_CURRENCIES.currency_code%TYPE;
BEGIN
    OPEN currencies_cur;
    LOOP
        FETCH currencies_cur INTO v_curr_code, v_curr_name;
        EXIT WHEN currencies_cur%NOTFOUND;
        DBMS_OUTPUT.PUT_LINE(v_curr_code || ',' || v_curr_name);
    END LOOP;
END;
\end{verbatim}
\item Read and display all the rows in the countries table where region = 5:
\begin{verbatim}
DECLARE
    CURSOR countries_cur IS SELECT
        country_name,national_holiday_name,national_holiday_date
    FROM WF_COUNTRIES WHERE REGION_ID = 5;
    v_country_name WF_COUNTRIES.country_name%TYPE;
    v_nat_holiday_name WF_COUNTRIES.national_holiday_name%TYPE;
    v_nat_holiday_date WF_COUNTRIES.national_holiday_date%TYPE;
BEGIN
    OPEN countries_cur;
    LOOP
        FETCH countries_cur INTO v_country_name, v_nat_holiday_name, v_nat_holiday_date;
        EXIT WHEN countries_cur%NOTFOUND;
        DBMS_OUTPUT.PUT_LINE(v_country_name || ',' || v_nat_holiday_name || ',' || v_nat_holiday_date);
    END LOOP;
END;
\end{verbatim}
\end{enumerate}
\item Three guidelines for fetching data from a cursor
\begin{enumerate}
\item Include the same number of variables in the INTO clause of the FETCH statement as columns in the SELECT statement
\item Match each variable to correspond to the columns position in the cursor definition.
\item Use \%TYPE to insure data types are compatible between variable and table.
\end{enumerate}
\end{enumerate}
\end{document}