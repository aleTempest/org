% Created 2023-11-26 Sun 20:39
% Intended LaTeX compiler: pdflatex
\documentclass[11pt]{article}
\usepackage[utf8]{inputenc}
\usepackage[T1]{fontenc}
\usepackage{graphicx}
\usepackage{longtable}
\usepackage{wrapfig}
\usepackage{rotating}
\usepackage[normalem]{ulem}
\usepackage{amsmath}
\usepackage{amssymb}
\usepackage{capt-of}
\usepackage{hyperref}
\usepackage[margin=0.5in]{geometry}
\author{Agustín Alejandro Mota Hinojosa}
\date{\today}
\title{Practice PL/SQL 5-2}
\hypersetup{
 pdfauthor={Agustín Alejandro Mota Hinojosa},
 pdftitle={Practice PL/SQL 5-2},
 pdfkeywords={},
 pdfsubject={},
 pdfcreator={Emacs 29.1 (Org mode 9.7)}, 
 pdflang={English}}
\begin{document}

\maketitle
\tableofcontents

\section{Vocabulary}
\label{sec:orgb8fa6ff}
\begin{enumerate}
\item Collection
\item INDEX BY TABLE
\item INDEX BY table of records
\end{enumerate}
\section{Try/Solve It}
\label{sec:orgb6d28a8}
\begin{enumerate}
\item PL/SQL Collections
\begin{enumerate}
\item is a named set of many occurrences of the same kind of data stored as a variable.
\item Collections
\begin{enumerate}
\item A list of all employees.
\item The population of all countries in Europe.
\item All the data stored in the employees table about specific employees.
\end{enumerate}
\item An INDEX BY table is based on a single field or a column of a table.
\item A INDEX BY table uses a TYPE to populate and an INDEX BY table of records uses
a composite TYPE.
\item The difference between t\textsubscript{pops} and v\textsubscript{pops}\textsubscript{tab} is that t\textsubscript{pops} is a TYPE and v\textsubscript{pops}\textsubscript{tab}
is a variable from that type. And v\textsubscript{pops}\textsubscript{tab} is a INDEX BY table because it's only
a column based field and not a composite TYPE.
\end{enumerate}
\item INDEX BY tables of countries in South America
\begin{enumerate}
\item INDEX BY table of countries in South America
\begin{verbatim}
DECLARE
    TYPE t_country_rec IS TABLE OF WF_COUNTRIES.COUNTRY_NAME%TYPE
        INDEX BY BINARY_INTEGER;
    v_country_rec_tab t_country_rec;
    v_count_id  BINARY_INTEGER := 0;
BEGIN
    FOR country_rec IN (
        SELECT country_id,country_name FROM WF_COUNTRIES
        WHERE region_id = 5 ORDER BY country_id)
        LOOP
        v_count_id := v_count_id + 1;
            v_country_rec_tab(v_count_id) := country_rec.COUNTRY_NAME;
        END LOOP;

END;
\end{verbatim}
\item Display de contents of de INDEX BY table
\begin{verbatim}
DECLARE
    TYPE t_country_rec IS TABLE OF WF_COUNTRIES.COUNTRY_NAME%TYPE
        INDEX BY BINARY_INTEGER;
    v_country_rec_tab t_country_rec;
    v_count_id  BINARY_INTEGER := 0;
    v_count BINARY_INTEGER := 0;
BEGIN
    FOR country_rec IN (
        SELECT country_id,country_name FROM WF_COUNTRIES
        WHERE region_id = 5 ORDER BY country_id)
        LOOP
        v_count_id := v_count_id + 1;
            v_country_rec_tab(v_count_id) := country_rec.COUNTRY_NAME;
        END LOOP;

    v_count := v_country_rec_tab.FIRST;
    LOOP
        IF v_country_rec_tab.EXISTS(v_count) THEN
            DBMS_OUTPUT.PUT_LINE(v_country_rec_tab(v_count));
            v_count := v_count + 1;
            EXIT WHEN v_count = v_country_rec_tab.LAST;
        END IF;
    end loop;
END;
\end{verbatim}
\item Only display the first and last elements of the INDEX BY table
\begin{verbatim}
DECLARE
    TYPE t_country_rec IS TABLE OF WF_COUNTRIES.COUNTRY_NAME%TYPE
        INDEX BY BINARY_INTEGER;
    v_country_rec_tab t_country_rec;
    v_count_id  BINARY_INTEGER := 0;
BEGIN
    FOR country_rec IN (
        SELECT country_id,country_name FROM WF_COUNTRIES
        WHERE region_id = 5 ORDER BY country_id)
        LOOP
        v_count_id := v_count_id + 1;
            v_country_rec_tab(v_count_id) := country_rec.COUNTRY_NAME;
        END LOOP;

    DBMS_OUTPUT.PUT_LINE('First element: ' || v_country_rec_tab(v_country_rec_tab.FIRST));
    DBMS_OUTPUT.PUT_LINE('Last element: ' || v_country_rec_tab(v_country_rec_tab.LAST));
\end{verbatim}
\end{enumerate}
\item INDEX BY table of records
\begin{enumerate}
\item Populate an INDEX BY table of records with EMPLOYEES data.
\begin{verbatim}
DECLARE
    TYPE t_emp_reg IS TABLE OF EMPLOYEES%ROWTYPE
        INDEX BY BINARY_INTEGER;
    v_emp_rec_tab t_emp_reg;
BEGIN
    FOR emp_reg IN (SELECT * FROM EMPLOYEES ORDER BY employee_id) LOOP
        v_emp_rec_tab(emp_reg.employee_id) := emp_reg;
    END LOOP;
END;
\end{verbatim}

\item Display the contends of the INDEX BY table of records
\end{enumerate}
\end{enumerate}
\begin{verbatim}
DECLARE
    TYPE t_emp_reg IS TABLE OF EMPLOYEES%ROWTYPE
        INDEX BY BINARY_INTEGER;
    v_emp_rec_tab t_emp_reg;
    v_count BINARY_INTEGER := 0;
BEGIN
    FOR emp_reg IN (SELECT * FROM EMPLOYEES ORDER BY employee_id) LOOP
        v_emp_rec_tab(emp_reg.employee_id) := emp_reg;
    END LOOP;
    v_count := v_emp_rec_tab.FIRST;
    LOOP
        IF v_emp_rec_tab.EXISTS(v_count) THEN
            DBMS_OUTPUT.PUT_LINE(v_emp_rec_tab(v_count).LAST_NAME);
        END IF;
        v_count := v_count + 1;
        EXIT WHEN v_count = v_emp_rec_tab.LAST;
    END LOOP;
END;
\end{verbatim}
\end{document}