% Created 2023-10-18 mié 15:57
% Intended LaTeX compiler: pdflatex
\documentclass[11pt]{article}
\usepackage[utf8]{inputenc}
\usepackage[T1]{fontenc}
\usepackage{graphicx}
\usepackage{longtable}
\usepackage{wrapfig}
\usepackage{rotating}
\usepackage[normalem]{ulem}
\usepackage{amsmath}
\usepackage{amssymb}
\usepackage{capt-of}
\usepackage{hyperref}
\usepackage{minted}
\usepackage[margin=0.5in]{geometry}
\usepackage[spanish, english]{babel}
\author{Agustín Alejandro Mota Hinojosa}
\date{\today}
\title{1-3 Creating PL/SQL Blocks}
\hypersetup{
 pdfauthor={Agustín Alejandro Mota Hinojosa},
 pdftitle={1-3 Creating PL/SQL Blocks},
 pdfkeywords={},
 pdfsubject={},
 pdfcreator={Emacs 29.1 (Org mode 9.7)}, 
 pdflang={English}}
\begin{document}

\maketitle
\tableofcontents

\section{Vocabulary}
\label{sec:orge6e193d}
\begin{enumerate}
\item Unnamed blocks of code not stored in the database and do not exist after
they are executed

\textbf{Anonymous Blocks}

\item A program that computes and returns a single value

\textbf{Subprogram declared as a function}

\item Named PL/SQL blocks that are stored in the database and can be declared as
procedures or functions

\textbf{Subprogram}

\item Software that checks and translates programs written in high-level
programming languages into binary code to execute

\textbf{Compiler}

\item A program that performs an action, but does not have to return a value

\textbf{Subprogram declared as a procedure}
\end{enumerate}
\section{Try / Solve it}
\label{sec:org113d4ae}
\begin{enumerate}
\item Complete the following chart defining the syntactical requirements for a
PL/SQL block:
\begin{center}
\begin{tabular}{lll}
 & Optional or Mandatory? & Describe\\[0pt]
\hline
\texttt{DECLARE} & optional & Contains declarations of all variables,\\[0pt]
 &  & constants, cursors, and user-defined\\[0pt]
 &  & exceptions that are referenced in the\\[0pt]
 &  & executable and exception sections.\\[0pt]
\texttt{BEGIN} & mandatory & Contains SQL statements to retrieve\\[0pt]
 &  & data from the database and PL/SQL\\[0pt]
 &  & statements to manipulate data in the\\[0pt]
 &  & block.\\[0pt]
\texttt{EXCEPTION} & optional & Specifies the actions to perform when\\[0pt]
 &  & errors and abnormal conditions arise in\\[0pt]
 &  & the executable section.\\[0pt]
\texttt{END;} & mandatory & Ends a \texttt{BEGIN} statement\\[0pt]
\end{tabular}
\end{center}

\item Which of the following PL/SQL blocks executes successfully? For the blocks
that fail, explain why they fail

\textbf{Option D executes correctly without any problems}

\begin{minted}[]{sql}
    DECLARE
        amount NUMBER(10);
    BEGIN
       DBMS_OUTPUT.PUT_LINE(amount);
    END;
\end{minted}

\item Fill in the blanks:

\begin{enumerate}
\item PL/SQL blocks that have no names are called \textbf{anonymous}

\item \texttt{FUNCTION} and \texttt{PROCEDURE} are named blocks and are stored in the database.
\end{enumerate}

\item In Application Express, create and execute a simple anonymous block that
outputs “Hello World.”
\begin{minted}[]{sql}
    BEGIN
        DBMS_OUTPUT.PUT_LINE('Hello World');
    END;
\end{minted}

\item Create and execute a simple anonymous block that does the following:
\begin{enumerate}
\item Declares a variable of datatype DATE and populates it with the date
that is six months from today

\begin{minted}[]{sql}
        DECLARE
            six_months_from_today date;
        BEGIN
            six_months_from_today := add_months(sysdate,6);
        END;
\end{minted}

\item Outputs “In six months, the date will be: <insert date>.”
\begin{minted}[]{sql}
        BEGIN
            DBMS_OUTPUT.PUT_LINE('In six months, the date will be ' ||
                                     six_months_from_today);
        END;
\end{minted}
\end{enumerate}
\end{enumerate}
\end{document}