% Created 2023-11-26 Sun 13:38
% Intended LaTeX compiler: pdflatex
\documentclass[11pt]{article}
\usepackage[utf8]{inputenc}
\usepackage[T1]{fontenc}
\usepackage{graphicx}
\usepackage{longtable}
\usepackage{wrapfig}
\usepackage{rotating}
\usepackage[normalem]{ulem}
\usepackage{amsmath}
\usepackage{amssymb}
\usepackage{capt-of}
\usepackage{hyperref}
\usepackage[margin=0.5in]{geometry}
\author{Agustín Alejandro Mota Hinojosa}
\date{\today}
\title{PL/SQL 6-4}
\hypersetup{
 pdfauthor={Agustín Alejandro Mota Hinojosa},
 pdftitle={PL/SQL 6-4},
 pdfkeywords={},
 pdfsubject={},
 pdfcreator={Emacs 29.1 (Org mode 9.7)}, 
 pdflang={English}}
\begin{document}

\maketitle
\tableofcontents

\section{Try It / Solve It}
\label{sec:orgd909e48}

\begin{enumerate}
\item Describe the benefit of using one or more parameters with a cursor.

You can parse custom values to the cursor

\item Write a PL/SQL block to display the country name and the area of each country in a chosen region. The region\textsubscript{id} should be passed to the cursor as a parameter. Test your block using two region\textsubscript{ids}: 5 (South America) and 30 (Eastern Asia). Do not use a cursor FOR loop.
\begin{verbatim}
DECLARE
    CURSOR cur_countries (p_regionid NUMBER) IS
        SELECT country_name, area FROM wf_countries
        WHERE region_id = p_regionid;
    v_countries_rec cur_countries%ROWTYPE;
    v_region_id NUMBER;
BEGIN
    v_region_id := 5;
    OPEN cur_countries(v_region_id);
    LOOP
        FETCH cur_countries INTO v_countries_rec;
        EXIT WHEN cur_countries%NOTFOUND;
        DBMS_OUTPUT.PUT_LINE(v_countries_rec.country_name || ' ' || v_countries_rec.area);
    END LOOP;
    CLOSE cur_countries;
    v_region_id := 30;
    OPEN cur_countries(v_region_id);
    LOOP
        FETCH cur_countries INTO v_countries_rec;
        EXIT WHEN cur_countries%NOTFOUND;
        DBMS_OUTPUT.PUT_LINE(v_countries_rec.country_name || ' ' || v_countries_rec.area);
    END LOOP;
    CLOSE cur_countries;
END;
\end{verbatim}

\item Modify your answer to question 2 to use a cursor FOR loop. You must still declare the cursor explicitly in the DECLARE section. Test it again using regions 5 and 30.
\begin{verbatim}
DECLARE
    CURSOR cur_countries (p_region_id NUMBER) IS (
        SELECT country_name,area
            FROM WF_COUNTRIES
        WHERE region_id = p_region_id
    );
    v_region_id NUMBER;
BEGIN
    v_region_id := 5;
    FOR rec IN cur_countries(v_region_id) LOOP
        DBMS_OUTPUT.PUT_LINE(rec.country_name || ' ' || rec.area);
    END LOOP;

    v_region_id := 30;
    FOR rec IN cur_countries(v_region_id) LOOP
        DBMS_OUTPUT.PUT_LINE(rec.country_name || ' ' || rec.area);
    END LOOP;
END;
\end{verbatim}

\item Modify your answer to question 3 to display the country\textsubscript{name} and area of each country in a chosen region that has an area greater than a specific value. The region\textsubscript{id} and specific area should be passed to the cursor as two parameters. Test your block twice using region\textsubscript{id} 5 (South America): the first time with area = 200000 and the second time with area = 1000000.
\begin{verbatim}
DECLARE
    CURSOR cur_countries (p_region_id NUMBER, p_area NUMBER) IS (
        SELECT country_name,area
            FROM WF_COUNTRIES
        WHERE region_id = p_region_id AND area > p_area
    );
    v_region_id NUMBER;
    v_area NUMBER;
BEGIN
    v_region_id := 5;
    v_area := 200000;
    FOR rec IN cur_countries(v_region_id,v_area) LOOP
        DBMS_OUTPUT.PUT_LINE(rec.country_name || ' ' || rec.area);
    END LOOP;

    v_region_id := 30;
    v_area := 1000000;
    FOR rec IN cur_countries(v_region_id,v_area) LOOP
        DBMS_OUTPUT.PUT_LINE(rec.country_name || ' ' || rec.area);
    END LOOP;
END;
\end{verbatim}

\item Modify your answer to question 4 to fetch and display two sets of countries in a single execution of the block. You should open and close the cursor twice, passing different parameter values to it each time. Before each set of output rows, display the message “Region: <region\textsubscript{id}> Minimum Area: <area>”., for example “Region: 5 Minimum Area: 200000”. Test your changes using (5, 200000) and (30, 500000).
\begin{verbatim}
DECLARE
    CURSOR cur_countries (p_regionid NUMBER, p_area NUMBER) IS
        SELECT country_name, area FROM wf_countries
        WHERE region_id = p_regionid AND area > p_area;
    v_countries_rec cur_countries%ROWTYPE;
    v_region_id NUMBER;
    v_area NUMBER;
BEGIN
    v_region_id := 5;
    v_area := 200000;
    OPEN cur_countries(v_region_id,v_area);
    LOOP
        FETCH cur_countries INTO v_countries_rec;
        EXIT WHEN cur_countries%NOTFOUND;
        DBMS_OUTPUT.PUT_LINE(v_countries_rec.country_name || ' ' || v_countries_rec.area);
    END LOOP;
    CLOSE cur_countries;
    v_region_id := 30;
    v_area := 500000;
    OPEN cur_countries(v_region_id,v_area);
    LOOP
        FETCH cur_countries INTO v_countries_rec;
        EXIT WHEN cur_countries%NOTFOUND;
        DBMS_OUTPUT.PUT_LINE(v_countries_rec.country_name || ' ' || v_countries_rec.area);
    END LOOP;
    CLOSE cur_countries;
END;
\end{verbatim}
\end{enumerate}
\end{document}