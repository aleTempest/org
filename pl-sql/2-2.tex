% Created 2023-10-20 vie 22:35
% Intended LaTeX compiler: pdflatex
\documentclass[11pt]{article}
\usepackage[utf8]{inputenc}
\usepackage[T1]{fontenc}
\usepackage{graphicx}
\usepackage{longtable}
\usepackage{wrapfig}
\usepackage{rotating}
\usepackage[normalem]{ulem}
\usepackage{amsmath}
\usepackage{amssymb}
\usepackage{capt-of}
\usepackage{hyperref}
\usepackage{minted}
\usepackage[margin=0.5in]{geometry}
\usepackage[spanish, english]{babel}
\author{Agustín Alejandro Mota Hinojosa}
\date{\today}
\title{2-2 Recognizing PL/SQL Lexical Units}
\hypersetup{
 pdfauthor={Agustín Alejandro Mota Hinojosa},
 pdftitle={2-2 Recognizing PL/SQL Lexical Units},
 pdfkeywords={},
 pdfsubject={},
 pdfcreator={Emacs 29.1 (Org mode 9.7)}, 
 pdflang={English}}
\begin{document}

\maketitle
\tableofcontents

\section{Vocabulary}
\label{sec:org3bedc90}
\begin{enumerate}
\item An explicit numeric, character string, date, or Boolean value that is not
represented by an identifier.

\textbf{Literal}

\item Symbols that have special meaning to an Oracle database.

\textbf{Escape characters}

\item Words that have special meaning to an Oracle database and cannot be used
as identifiers.

\textbf{Reserved Words}

\item Describe the purpose and use of each code segment and are ignored by
PL/SQL.

\textbf{Comments}

\item Building blocks of any PL/SQL block and are sequences of characters
including letters, digits, tabs, returns, and symbols.

\textbf{Identifiers}

\item A name, up to 30 characters in length, given to a PL/SQL object.

\textbf{Identifiers}
\end{enumerate}
\section{Try It / Solve It Questions}
\label{sec:orge504f3a}
\begin{enumerate}
\item Identify each of the following identifiers as valid or invalid. If invalid,
specify why.
\begin{enumerate}
\item \texttt{Today}
Invalid (Identifiers cannot start with or contain equal signs).
\item \texttt{Last name}
Invalid (Identifiers cannot contain spaces).
\item \texttt{today’s\_date}
Invalid (Identifiers cannot contain apostrophes).
\item \texttt{number\_of\_days\_in\_february\_this\_}
Invalid (Identifiers cannot end with an underscore followed by an equal sign).
\item \texttt{year}
Valid
\item \texttt{Isleap\$year}
Valid
\item \texttt{\#number}
Invalid (Identifiers cannot start with a pound sign).
\item \texttt{NUMBER\#}
Invalid (Identifiers cannot contain pound signs).
\item \texttt{Number1to7}
Valid
\end{enumerate}

\item Identify the reserved words in the following list.
\begin{center}
\begin{tabular}{ll}
Word & Reserved\\[0pt]
\hline
create & X\\[0pt]
make & \\[0pt]
table & X\\[0pt]
seat & \\[0pt]
alter & X\\[0pt]
rename & X\\[0pt]
row & \\[0pt]
number & \\[0pt]
web & \\[0pt]
\end{tabular}
\end{center}
\item What kind of lexical unit (for example Reserved word, Delimiter, Literal,
Comment) is each of the following?
\begin{center}
\begin{tabular}{ll}
Value & Lexical Unit\\[0pt]
\hline
\texttt{SELECT} & Reserved Word\\[0pt]
\texttt{:=} & Delimeter\\[0pt]
\texttt{'TEST'} & Literal\\[0pt]
\texttt{FALSE} & Literal\\[0pt]
\texttt{-{}-{} new process} & Comment\\[0pt]
\texttt{FROM} & Reserved Word\\[0pt]
/* select the country with the high- & Comment\\[0pt]
est elevation */ & \\[0pt]
\texttt{v\_test} & Identifier\\[0pt]
\texttt{4.09} & Literal\\[0pt]
\end{tabular}
\end{center}
\end{enumerate}
\end{document}