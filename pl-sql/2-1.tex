% Created 2023-10-20 vie 22:18
% Intended LaTeX compiler: pdflatex
\documentclass[11pt]{article}
\usepackage[utf8]{inputenc}
\usepackage[T1]{fontenc}
\usepackage{graphicx}
\usepackage{longtable}
\usepackage{wrapfig}
\usepackage{rotating}
\usepackage[normalem]{ulem}
\usepackage{amsmath}
\usepackage{amssymb}
\usepackage{capt-of}
\usepackage{hyperref}
\usepackage{minted}
\usepackage[margin=0.5in]{geometry}
\usepackage[spanish, english]{babel}
\author{Agustín Alejandro Mota Hinojosa}
\date{\today}
\title{2-1 Using Variables in PL/SQL}
\hypersetup{
 pdfauthor={Agustín Alejandro Mota Hinojosa},
 pdftitle={2-1 Using Variables in PL/SQL},
 pdfkeywords={},
 pdfsubject={},
 pdfcreator={Emacs 29.1 (Org mode 9.7)}, 
 pdflang={English}}
\begin{document}

\maketitle
\tableofcontents

\section{Vocabulary}
\label{sec:org8265613}
\begin{enumerate}
\item Used for storage of data and manipulation of stored values.

\textbf{Variables}

\item Values passed to a program by a user or by another program to customize
the program.

\textbf{Parameters}
\end{enumerate}
\section{Try It / Solve It}
\label{sec:org0a22489}
\begin{enumerate}
\item Fill in the blanks.
\begin{enumerate}
\item Variables can be assigned to the output of a \textbf{query}
\item Variables can be assigned values in the \textbf{DECLARE} section of a PL/SQL block.
\item Variables can be passed as \textbf{parameters} to subprograms.
\end{enumerate}

\item Identify valid and invalid variable declaration and initialization:
\begin{minted}[]{sql}
    number_of_copies PLS_INTEGER; -- valid
    printer_name CONSTANT VARCHAR2(10); -- valid
    by_when DATE := SYSDATE+1;
\end{minted}

The statement:
\begin{minted}[]{sql}
    deliver_to VARCHAR2(10) := Johnson; -- invalid
\end{minted}
is invalid, the correct statement should look like this:

\begin{minted}[]{sql}
    deliver_to VARCHAR2(10) := 'Johnson'; -- valid
\end{minted}

\item Examine the following anonymous block and choose the appropriate statement.
\begin{minted}[]{sql}
    DECLARE
        fname VARCHAR2(25);
        lname VARCHAR2(25) DEFAULT 'fernandez';
    BEGIN
        DBMS_OUTPUT.PUT_LINE(fname || ' ' || lname);
    END;
\end{minted}

\textbf{B. The block will give an error because the fname variable is used without initializing.}

\item In Application Express:
\begin{enumerate}
\item Create the following function:
\begin{minted}[]{sql}
        CREATE FUNCTION num_characters (p_string IN VARCHAR2)
        RETURN INTEGER AS
            v_num_characters INTEGER;
        BEGIN
        SELECT LENGTH(p_string) INTO v_num_characters
            FROM dual;
        RETURN v_num_characters;
        END;
\end{minted}
\item Create and execute the following anonymous block:
\begin{minted}[]{sql}
       DECLARE
           v_length_of_string INTEGER;
       BEGIN
           v_length_of_string := num_characters('Oracle Corporation');
           DBMS_OUTPUT.PUT_LINE(v_length_of_string);
       END;
\end{minted}
\end{enumerate}
\item Write an anonymous block that uses a country name as input and prints
the highest and lowest elevations for that country. Use the COUNTRIES
table. Execute your block three times using Unit- ed States of America,
French Republic, and Japan.
\begin{minted}[]{sql}
    DECLARE
        v_country_name VARCHAR2(100) := 'United States of America';
        v_highest_elevation NUMBER;
        v_lowest_elevation NUMBER;
    BEGIN
        SELECT MAX(ELEVATION), MIN(ELEVATION)
        INTO v_highest_elevation, v_lowest_elevation
        FROM COUNTRIES
    WHERE COUNTRY_NAME = v_country_name;
        DBMS_OUTPUT.PUT_LINE('Highest Elevation: ' || v_highest_elevation);
        DBMS_OUTPUT.PUT_LINE('Lowest Elevation: ' || v_lowest_elevation);
    END;
\end{minted}
\end{enumerate}
\end{document}