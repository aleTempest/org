% Created 2023-11-26 Sun 13:31
% Intended LaTeX compiler: pdflatex
\documentclass[11pt]{article}
\usepackage[utf8]{inputenc}
\usepackage[T1]{fontenc}
\usepackage{graphicx}
\usepackage{longtable}
\usepackage{wrapfig}
\usepackage{rotating}
\usepackage[normalem]{ulem}
\usepackage{amsmath}
\usepackage{amssymb}
\usepackage{capt-of}
\usepackage{hyperref}
\usepackage[margin=0.5in]{geometry}
\author{Agustín Alejandro Mota Hinojosa}
\date{\today}
\title{PL/SQL 6-2}
\hypersetup{
 pdfauthor={Agustín Alejandro Mota Hinojosa},
 pdftitle={PL/SQL 6-2},
 pdfkeywords={},
 pdfsubject={},
 pdfcreator={Emacs 29.1 (Org mode 9.7)}, 
 pdflang={English}}
\begin{document}

\maketitle
\tableofcontents

\section{Vocabulary}
\label{sec:orgb8ec2a9}

CURSOR Declared by the programmer for queries that return more than one row
RECORD An attribute that processes an exact number of rows or counts the number of rows fetched in a loop
CLOSE Disables a cursor, releases the context area, and undefines the active set
\section{Try It / Solve It}
\label{sec:orgc1fe2c4}

\begin{enumerate}
\item In your own words, explain the advantage of using \%ROWTYPE to declare a record structure based on a cursor declaration.

It's like a cursor that contains all the columns of a table.

\item Write a PL/SQL block to read through rows in the countries table for all countries in region 5 (South America region). For each selected country, display the country\textsubscript{name}, national\textsubscript{holiday}\textsubscript{date}, and national\textsubscript{holiday}\textsubscript{name}. Use a record structure to hold all the columns selected from the countries table.
\begin{verbatim}
DECLARE
    TYPE t_country_reg IS TABLE OF WF_COUNTRIES%ROWTYPE
        INDEX BY BINARY_INTEGER;
BEGIN
    FOR country_rec IN (SELECT * FROM WF_COUNTRIES) LOOP
        DBMS_OUTPUT.PUT_LINE(country_rec.COUNTRY_NAME || ' ' || country_rec.COUNTRY_NAME ||
                             ' ' || country_rec.NATIONAL_HOLIDAY_DATE || ' ' || country_rec.NATIONAL_HOLIDAY_NAME);
    END LOOP;
END;
\end{verbatim}

\item For this exercise, you use the employees table. Create a PL/SQL block that fetches and displays the six employees with the highest salary. For each of these employees, display the first name, last name, job id, and salary. Order your output so that the employee with the highest salary is displayed first. Use \%ROWTYPE and the explicit cursor attribute \%ROWCOUNT.
\begin{verbatim}
DECLARE
    TYPE t_emp_reg IS TABLE OF EMPLOYEES%ROWTYPE
        INDEX BY BINARY_INTEGER;
BEGIN
    FOR emp_rec IN (SELECT * FROM EMPLOYEES ORDER BY salary FETCH FIRST 6 ROWS ONLY) LOOP
        DBMS_OUTPUT.PUT_LINE(emp_rec.FIRST_NAME || ' ' || emp_rec.LAST_NAME || ' ' || emp_rec.SALARY);
    END LOOP;
END;
\end{verbatim}

\item Look again at the block you created in question 3. What if you wanted to display 21 employees instead of 6? There are only 20 rows in the employees table. What do you think would happen?

You would need to modify the number of rows directly.

\item In real life we would not know how many rows the table contained. Modify your block from question 3 so that it will exit from the loop when either 21 rows have been fetched and displayed, or when there are no more rows to fetch. Test the block again.
\begin{verbatim}
DECLARE
    TYPE t_emp_reg IS TABLE OF EMPLOYEES%ROWTYPE
        INDEX BY BINARY_INTEGER;
BEGIN
    FOR emp_rec IN (SELECT * FROM EMPLOYEES ORDER BY salary) LOOP
        DBMS_OUTPUT.PUT_LINE(emp_rec.FIRST_NAME || ' ' || emp_rec.LAST_NAME || ' ' || emp_rec.SALARY);
    END LOOP;
END;
\end{verbatim}
\end{enumerate}
\end{document}